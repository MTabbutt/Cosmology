
\documentclass{article}


%%% Use these packages:
\usepackage{geometry}
\usepackage{setspace}
\usepackage{soul}
\usepackage{color}
\usepackage{amsmath}


%%% Set things:
 \geometry{top = .75in, bottom=.75in, left=.75in, right=.75in}
 
 
 %%% Make the title section:
\title{ {\bf Theory Walk Through for Cross-Correlating Redshift-Free Standard Candles} \\
\vspace{10pt}
\LARGE Mukherjee and Wandelt (2018) Paper}
\date{April 2020}
\author{Megan Tabbutt}



\begin{document}

\doublespacing
\maketitle



\fontsize{11}{12}


% % % % % % % % % % % % % % % % % % % % % % % % 
\section{Abstract:}

LSST will supply up to $10^6$ supernovae (SNe) to constrain dark energy through the distance - redshift (D$_L$ - z) test. Obtaining spectroscopic SN redshifts (spec-zs) is unfeasible; alternatives are suboptimal and may be biased. We propose a powerful multi-tracer generalization of the Alcock- Paczynski test that pairs redshift-free distance tracers and an overlapping galaxy redshift survey. Cross-correlating $5\times 10^4$ redshift-free SNe with DESI or Euclid outperforms the classical D$_L$L - z test with spec-zs for all SN. Our method also applies to gravitational wave sirens or any redshift-free distance tracer.



% % % % % % % % % % % % % % % % % % % % % % % % 
\section{Introduction:}

\begin{itemize}

  \item Accurate trace of the expansion history, e.g. through the D$_L$ - z relation, is one of the foremost goals of current and next-generation surveys such as SDSS-IV, DES, DESI, EUCLID, LSST, and WFIRST
  
  \item LSST going to generate large type Ia SN sample at a rate of $\ge 10^4 yr^{-1}$. Already now, the $\ge 10^3 yr^{-1}$ SNe being observed over a wide redshift range make it impossible to obtain time-consuming spectroscopic follow-up for every SN, (the traditional approach underlying the success of cosmology with standard candles over the last two decades). 
  
  \item Alternative approach combines photometric types with spectroscopic redshift measurements of the presumed SN host galaxy,  Errors may lead to biases and loss of information in the inferred cosmological parameters.
  
  \item Future galaxy redshift surveys (DESI, EUCLID)  going to measure tens of millions of galaxy redshifts over large fractions of sky. It is realistic to expect a galaxy catalog with $10^7$ spectroscopic redshifts overlapping SN data sets over a wide redshift range on the time scale of LSST.
  
\end{itemize}

% % % % % % % % % % % % % % % % % % % % % % % % 
\section{Idea:}

Propose a new method to infer cosmological parameters accurately from distance tracers (e.g. SNe) without redshifts:
 
 \begin{itemize}

  \item Exploit  fact that both distance tracers and galaxies are tracers of the matter density, and therefore spatially correlated through the underlying matter field
  
  \item Can tightly constrain cosmology by maximizing the spatial cross-correlation of overlapping distance catalog and redshift catalogs
  
  \item Approach shows a classical cosmological test in a new light as a limit of a multi-tracer generalization of the Alcock-Paczynski (A-P) test
  
  \item Particular feature of this cross-correlation approach is its robustness to both data systematics and modeling assumptions
       
\end{itemize}


% % % % % % % % % % % % % % % % % % % % % % % % 
\section{Set-Up:}

% % % % % % % % % %
\subsection{Apparent Magnitude:}

Constrain Luminosity Distance, $D_L$ through relationship between apparent, $m$, and absolute, $M$, magnitudes calibrated from light curves: 

\begin{equation}
m = 5 log_{10}(\frac{D_{L}(z)}{pc}) + M - 5
\end{equation}

- Get light-curves from photometric observations, then calibrate the relationship and get $D_L$. For some data might already have this, like for DES-Sn.

- Equation is a definition of $D_L$, which is the thing you want to get and carry forward. 

- \hl{$m$ is the apparent magnitude, this comes from observations, at what point in the light curve do you pick for m ???} 

- $M$ is the absolute magnitude, this is gotten from fitting the light curve 

\clearpage
% % % % % % % % % %
\subsection{Luminosity Distance:}

The luminosity distance is related to the cosmological model and redshift through this equation:

\begin{equation}
D_{L}(z) = \frac{c}{H}(1 + z) \int_{0}^{z} \frac{dz'}{\sqrt{\mathcal{E}(z)}}
\end{equation}

\begin{equation}
\mathcal{E}(z) \equiv \Omega_m(1+z')^3 + \Omega_{de} \exp(3 \int^z_0 d \; ln(1+z')(1+\omega (z')))
\end{equation}

- Assume a flat universe ($\Omega_K = 0$, or $\Omega_{de} = 1 - \Omega_m$) with dark energy equation of state: $\omega(z) = \omega_0 + \omega_a(z/(1+z))$


% % % % % % % % % %
\subsection{Isotropic Two-Point Correlation Function:}

At a seperation, $r$, between two tracers ($x$ and $y$) of the density fluctuations, e.g. $1+\delta_x(\bf{s}) = \rho_x(\bf{s})/\bar{\rho}$, with respect to the background density, $\bar{\rho}$, can be written as: 

\begin{equation}
\xi^{iso}_{x-y}(r) = \frac{1}{2 \pi^2} \int k^2 \; dk \; b_x(z) \; b_y(z) \; P(k) \; j_0(kr) \; e^{-k^2/k^2_{max}}
\end{equation}

- $P(k)$ is the non-linear power spectrum obtained from the ensemble average of the density fluctuations in the Fourier domain for wavenumber $\bf{k}$

- $j_0(kr)$ is the spherical Bessel function

- $b_x = \delta_x/\delta_{dm}$ is the bias of the tracer x WRT Dark Matter - Things Ross has talked about before

- The cutoff $k_{max}$ is introduced for a numerical convergence at high $k$, to avoid the oscillatory behavior of $j_0(kr)$

- For galaxies: $x = g$, with galaxy bias $b_g \approx 1.6$,

- For SNe: $x = sn$ and $b_{sn} = \delta_{sn}/ \delta_{sdm}$. While $b_{sn}$ is uncertain there are studies which indicate that the $b_{sn}$ may exceed $b_{g}$ by around 60\%. We will conservatively take $b_{sn} \approx 1.6$ as a fiducial value

% % % % % % % % % %
\subsection{Red-Shift Space Distortions:}

RSD`s will affect galaxy positions in redshift space (not SNe because no red-shift label for us). Need to add a single Kaiser factor to the galaxy-SN cross-power spectrum: 

\begin{equation}
P_{gs}(k) = (1 + f \mu^2_k / b_g) P_k
\end{equation}

- $f \equiv d \; ln G / d \; ln a$ is dimensionless growth rate
- $G$ is growth factor, $\mu_k$ is the cosine of the angle between the Fourier modes and the line of sight 

Keith Suggestions: 

\begin{itemize}
  \item Kaiser 1987 - this seems to be the seminar work in the field: \em{Clustering in real space and in red shift space - Kaiser, 1987}
  
  \item Hamilton 1992 - fairly technical, similar in spirit to the Kaiser 1987: \em{Measuring Omega and the Real Correlation Function from the Redshift Correlation Function - Hamilton, 1991}
  
  \item Percival et al. 2011 - more accessible review article: \em{Redshift-Space Distortions - Percival et al., 2011}
  
\end{itemize}

% % % % % % % % % %
\subsection{Anisotropic Cross Correlation Function:}


\begin{equation}
\xi_{g-sn}(\boldsymbol{r}) = (1 + \frac{f}{3 b_g}) \;  \xi^{iso}_{g-sn}(r) \; \mathcal{P}_0(\mu_r) + \frac{2 f}{3 b_g} \;  (\xi^{iso}_{g-sn}(r) - \bar{\xi}_{gs}(r)) \;  \mathcal{P}_2(\mu_r)
\end{equation}

- $\bar{\xi}_{gs}(r) = \frac{3}{r^3}\int^r_0 \xi^{iso}_{g-sn}(s)s^2 \; ds$

- $\mu_r$ is the cosine of the angle between the line of sight and $\textbf{r}$

- $\mathcal{P}_{\mathcal{\ell}}(\mu_r)$ is the $\mathcal{\ell}$th order Legendre Polynomial. 

- Anisotropic $g-g$ and $sn-sn$ autocorrelations take on analogous forms.  


% % % % % % % % % %
\subsection{Giant Equation:}

\begin{equation}
\xi_{g-sn}(\textbf{r}) = 
(1 + \frac{f}{3 b_g}) \;  \xi^{iso}_{g-sn}(r) \; \mathcal{P}_0(\mu_r) + 
\frac{2 f}{3 b_g} \;  (\xi^{iso}_{g-sn}(r) -\frac{3}{r^3}\int^r_0 \xi^{iso}_{g-sn}(s)s^2 \; ds) \;  \mathcal{P}_2(\mu_r)
\end{equation}

\begin{equation}
\xi^{iso}_{x-y}(r) = \frac{1}{2 \pi^2} \int k^2 \; dk \; b_x(z) \; b_y(z) \; P(k) \; j_0(kr) \; e^{-k^2/k^2_{max}}
\end{equation}


% % % % % % % % % %% % % % % % % % % %% % % % % % % % % %
\section{Method:}

Consider both galaxies and SNe as tracers of $\delta_{dm}$ in comoving coordinates. Galaxies observed in redshift space and SNe in $D_L$ space. 

Model these observed galaxy and SNe over-densities, $\boldsymbol{\delta}_{g, sn}$ as correlated Gaussian random fields. Then log-likelihood for: $\boldsymbol{\theta} \equiv \{\Omega_m, \omega_0, \omega_a, H_0\}$ becomes: 

\begin{equation}
- 2 \; \mathcal{L}_{full}(\boldsymbol{\delta}_g, \boldsymbol{\delta}_{sn} | \boldsymbol{\theta}) = \left(
\begin{array}{c} 
\boldsymbol{\delta}_g \\
\boldsymbol{\delta}_{sn} \\ 
\end{array} 
\right) ^T  \boldsymbol{\Xi}^{-1} \left(
\begin{array}{c} 
\boldsymbol{\delta}_g \\
\boldsymbol{\delta}_{sn} \\ 
\end{array} 
\right) \; + \; \text{ln} | \boldsymbol{\Xi} | 
\end{equation}

- The Covariance matrix, $\boldsymbol{\Xi}(\boldsymbol{\theta})$ is a block matrix of form:

\begin{equation}
\boldsymbol{\Xi}(\boldsymbol{\theta}) = \left(
\begin{array}{cc} 
\boldsymbol{Z}^T(\boldsymbol{\theta}) \boldsymbol{\xi}_{g-g} \boldsymbol{Z}(\boldsymbol{\theta})  & \boldsymbol{Z}^T(\boldsymbol{\theta}) \boldsymbol{\xi}_{g-sn} \boldsymbol{D}(\boldsymbol{\theta})\\
\boldsymbol{D}^T(\boldsymbol{\theta}) \boldsymbol{\xi}_{g-sn}^T \boldsymbol{Z}(\boldsymbol{\theta}) & \boldsymbol{D}^T(\boldsymbol{\theta}) \boldsymbol{\xi}_{sn-sn} \boldsymbol{D}(\boldsymbol{\theta})\\ 
\end{array} 
\right)
\end{equation}

- The $\boldsymbol{\xi}$ are computed as outlines above. 

- Parameter dependence in $\boldsymbol{\Xi}$ enters only through the mappings, $\boldsymbol{Z}$ and $\boldsymbol{D}$, from the comoving coordinates to $z$ and $D_L$ respectively. 

- The form of the $\boldsymbol{\xi}$ are assumed to be fixed to a fiducial cosmology, and the optimal parameters will choose mappings $\boldsymbol{Z}$ and $\boldsymbol{D}$ that best re-map the forms of the correlation functions into the data space. 



\end{document}




















